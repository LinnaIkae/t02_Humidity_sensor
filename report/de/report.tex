% !TeX program = lualatex
\documentclass[12pt,fleqn,parskip=half,twoside,toc=index,headings=small,a4paper]{scrreprt}
%\usepackage[utf8]{inputenc}
\usepackage[T1]{fontenc}
\usepackage[  
    vmarginratio=1:2, %Verhältnis der oben/unten Seitenränder zur automatischen Berechnung
    paper=a4paper,
    %left=3cm,
    %right=3cm,
    inner=3.5cm, % mittlerer Rand
    outer=2.5cm, % äußerer Rand
    top=2.0cm,
    bottom=2.5cm,
    %marginparwidth=2.3cm, % Breite des Marginpars
    includeheadfoot, % Kopfzeile in Berechnung einbeziehen
    %includemp % Marginpar in die Berechnung mit einbeziehen
    %showframe=true,
]{geometry}
\usepackage{tabularx}
\usepackage{booktabs}
\usepackage[ngerman]{babel}
\usepackage{mathpazo}
\usepackage[euler-digits]{eulervm}

\usepackage[no-math]{fontspec} %no-math
%\usepackage{fontspec}
\usepackage{amsmath}
%\usepackage[math-style=upright]{unicode-math}
%\setmathfont{TeX Gyre Pagella Math}
%\setmathfont{Neo Euler}
%\%setmathfont[math-style=ISO,bold-style=ISO,vargreek-shape=TeX]{Libertinus Math}
%\defaultfontfeatures{SmallCapsFeatures={Renderer=Basic}}
\setsansfont[Path =../fonts/,
UprightFont = *-Regular,
BoldFont = *-Bold,
ItalicFont = *-Italic,
BoldItalicFont = *-BoldItalic
]{URWClassico}
 
%\setmainfont[Path =./fonts/,
%UprightFont = *-regular,
%BoldFont = *-bold,
%ItalicFont = *-italic,
%BoldItalicFont = *-bolditalic]{texgyrepagella}

\setmainfont[Path =../fonts/,
UprightFont = *,
BoldFont = *b,
ItalicFont = *i,
BoldItalicFont = *bi]{pala}

\setmonofont[Path = ../fonts/,
UprightFont = *-Regular,
BoldFont = *-Bold,
AutoFakeSlant,
BoldItalicFeatures={FakeSlant}
]{Inconsolata}

\usepackage{MnSymbol}

\usepackage[tracking=true]{microtype}
\SetTracking[no ligatures]{encoding = *, shape = sc*}{150}
\usepackage[
pdfencoding=auto,
colorlinks=true,
linkcolor=black,
citecolor=black,
filecolor=black,
urlcolor=black,
bookmarks=true,
bookmarksopen=true,
bookmarksopenlevel=3,
plainpages=false,
pdfpagelabels=true,
pdftitle={Praktikum Mikrocontroller},
pdfauthor={},
pdfsubject={}, % subject of the document
pdfkeywords={}]{hyperref}
\usepackage{mathtools}
\usepackage{setspace}
\spacing{1.2}
\usepackage{enumitem}
%\usepackage{rotating}
\usepackage{booktabs}
\usepackage{pgfplots}
\usetikzlibrary{pgfplots.groupplots} 
\usetikzlibrary{plotmarks}
\usepackage{tikz-3dplot}
\usetikzlibrary{calc}
\usepackage{polynom}
\polyset{style=C, div=:,vars=x}
\usepackage{wrapfig}
\usepackage[locale=DE]{siunitx}
\DeclareSIUnit{\promille}{\text{\textperthousand}}
%\DeclareSIUnit{\percent}{\text{\textpercent}}
\DeclareSIUnit{\dBm}{dBm}
\sisetup{
	math-rm=\mathrm,
	mode=math,
	unitmode=text,
    %detect-all,               %% Benutze gleiche Schriftarten wie im Text
    group-digits=true,          %% Zifferngruppierung an/aus
    group-separator=\, ,        %% Zeichen für Zifferngruppierung
    group-minimum-digits=5     %% Ziffern ab # Ziffern gruppieren
}
\usepackage{relsize}
\usepackage{cancel}
\usepackage{nicefrac}
\usepackage{xcolor}
\usepackage{tikz}
\usepackage{lscape}
\usepackage{here}
\usepackage{selnolig}
\nolig{Haupttask}{Haupt|task}
\nolig{Starttemperatur}{Start|temperatur}
\usepackage{scrpage2} 
\setheadsepline{0.5pt}
\clearscrheadings
\clearscrplain                  
\pagestyle{scrheadings}
\automark[section]{chapter}
%Nun die Befehle für die Kopf- und Fusszeileneinstellung. Der Befehl ist immer für beide Seitenstile (plain,headings) und hat immer das Format:
%\Position[scrplain]{scrheadings}
\ohead[]{\rightmark}
\ofoot[\pagemark]{\pagemark}
\renewcommand*{\headfont}{\normalfont\sffamily}
\renewcommand*{\footfont}{\normalfont\sffamily}
\renewcommand*{\pnumfont}{\normalfont\sffamily}

%\usepackage{mathtools}
\usetikzlibrary{positioning,shadows,backgrounds}
% Wurzel mit Haken hinten (Dieser Hack hat eine Größenordnung nationaler Verschuldung)
\usepackage{letltxmacro}
\LetLtxMacro{\oldsqrt}{\sqrt}
\renewcommand\sqrt[2][]{\mathpalette\DHLhksqrtA{{#1}{#2}}}
\def\DHLhksqrtA#1#2{\setbox0=\hbox{$#1\DHLhksqrtB#2$}\dimen0=\ht0
\advance\dimen0-0.15\ht0
%0.15 ist das Mass fuer die Hakenlaenge, relativ zum Inhalt der Wurzel
\setbox2=\hbox{\vrule height 0.98\ht0 depth -\dimen0}%
{\box0\lower0.4pt\box2}}
\def\DHLhksqrtB#1#2{\def\a{#1}\def\b{}\ifx\a\b\oldsqrt{#2\,}\else\oldsqrt[#1]{#2\,}\fi}

\newcommand{\remark}[1]{{\color{red}#1}}

\usepackage[tocindentauto]{tocstyle}
\usetocstyle{KOMAlike}

\newsavebox{\myendhook}  % for the tabulars
\makeatletter
\def\tagform@#1{{(\maketag@@@{\ignorespaces#1\unskip\@@italiccorr)}%
		\makebox[0pt][r]{%  after  the  equation  number
			\makebox[0.4\textwidth][l]{\usebox{\myendhook}}%
		}%
		\global\sbox{\myendhook}{}%  clear  box  content
	}%
}
\makeatother

\usepackage{ncccomma} %Komma als Dezimaltrennzeichen
\usepackage{url}

\addtokomafont{disposition}{\boldmath}
\newcommand{\abs}[1]{\ensuremath{\left\vert#1\right\vert}}
\usepackage[europeanresistors]{circuitikz}

\definecolor{dkgreen}{rgb}{0,0.6,0}
\definecolor{gray}{rgb}{0.5,0.5,0.5}
\definecolor{mauve}{rgb}{0.58,0,0.82}
\usepackage{listings}
\lstdefinestyle{myC}{ %
	columns=fullflexible,
  language=C,                  % the language of the code
  basicstyle=\footnotesize\ttfamily\color{black},       % the size of the fonts that are used for the code
  numbers=left,                   % where to put the line-numbers
  numberstyle=\tiny\color{black},  % the style that is used for the line-numbers
  stepnumber=5,firstnumber=1,numberfirstline,
                                  % will be numbered
  numbersep=5pt,                  % how far the line-numbers are from the code
  backgroundcolor=\color{white},  % choose the background color. You must add \usepackage{color}
  showspaces=false,               % show spaces adding particular underscores
  showstringspaces=false,         % underline spaces within strings
  showtabs=false,                 % show tabs within strings adding particular underscores
  frame=none,                   % adds a frame around the code
  rulecolor=\color{black},        % if not set, the frame-color may be changed on line-breaks within not-black text (e.g. commens (green here))
  tabsize=4,                      % sets default tabsize to 2 spaces
  captionpos=b,                   % sets the caption-position to bottom
  breaklines=true,                % sets automatic line breaking
  breakatwhitespace=true,        % sets if automatic breaks should only happen at whitespace
  title=\lstname,                 % show the filename of files included with \lstinputlisting;
                                  % also try caption instead of title
  keywordstyle=\bfseries\color{black},          % keyword style
  commentstyle=\itshape\color{gray},       % comment style
  stringstyle=\color{gray},         % string literal style
  escapeinside={\%*}{*)},            % if you want to add a comment within your code
  morekeywords={uint8_t,uint16_t,uint32_t},               % if you want to add more keywords to the set
  lineskip={1pt}
}


\usepackage{comment}

\begin{document}
	\begin{titlepage}
\begin{center}
{\huge \bfseries Microcomputers Lab\\}%[0.4cm]
\vfill
{\large \textsc{Report}}\\[9pt]
\vfill
{\large IoT}\\[9pt]
{\normalsize WS 2017/18}\\[9pt]
\vfill
\begin{minipage}{0.5\textwidth}
\normalsize
\begin{flushleft}
\begin{tabular}{@{}ll@{}} 
\textbf{Team 00} & \textbf{E-Mail}\\
Anna Musterfrau & \href{mailto:anna.musterfrau@uni-ulm.de}{anna.musterfrau@uni-ulm.de}\\
Max Mustermann & \href{mailto:max.mustermann@uni-ulm.de}{max.mustermann@uni-ulm.de}
\end{tabular}\\
\end{flushleft}
\end{minipage}
\hfill
\begin{minipage}{0.4\textwidth}
\normalsize
\begin{flushright}
\textbf{Tutor} \\
Tutor Machtalles
\end{flushright}
\end{minipage}
\vfill
{\normalsize \today}
\end{center}
\end{titlepage}
	\microtypesetup{protrusion=false}
	\tableofcontents
	\microtypesetup{protrusion=true}
	%%%%%%%%%%%%%%%%%%%%%%%%%%%%%%%%%%%%%%%%%%%
	\raggedbottom
	\chapter{Aufgabenstellung}
	Im Rahmen des Mikrocomputer-Praktikums soll mit \textsmaller{PWM} und \textsmaller{LED}s ansteuern.
%	
	\section{Peripherie}
	Von der \textsmaller{MCU} sind folgende Aktoren
	\begin{itemize}[noitemsep]
		\item ...
		\item ...
		\item ...
	\end{itemize}
	zu steuern.
	
	Als Sensoren sind
	\begin{itemize}[noitemsep]
		\item ...
		\item ...
	\end{itemize}
	auszuwerten.
	%%%%%%%%%%%%%%%%%%%%%%%%%%%%%%%%%%%%%%
	\chapter{Schaltungsentwurf}
	\section{Blockdiagramm}
	\begin{figure}[h]
		\centering
		\tikzstyle{block} = [draw, rectangle, 
		minimum height=3em, minimum width=6em, very thick]
		
		% The block diagram code is probably more verbose than necessary
		\begin{tikzpicture}[auto, node distance=2cm,>=latex']
		
		%\draw[help lines,step=5mm,gray!20] (0,0) grid (15,10);
		\node[block] (mcu) at (5,3) {CC2650};
		
		\node[block, above right = 0.5cm and 0.5cm of mcu] (L298) {H-Brücke};
		\node[block, above = 0.6cm of L298] (shu) {Shunt};
		\node[block, above = 0.6cm of shu] (mot) {\textsmaller{DC}-Motor};
		
		\node[block, left = 0.5cm of shu] (opa) {OpAmp};
		\node[block, left = 1cm of mcu] (reg) {Schieberegister};	
		
		\node[block, below = 0.6cm of reg] (led) {LEDs};
		\node[block, below = 0.6cm of mcu] (taster) {Taster};
		
		\node[block, above = 0.6cm of reg] (7Se) {7 Segment};
		
		
		\draw[thick, ->, shorten >=1mm,shorten <=1mm] (mcu.east) -| (L298.south) node [near start, xshift=3mm] {\small PWM};
		
		\draw[thick, ->, shorten >=1mm,shorten <=1mm] (mcu.west) -- (reg) node [near start, xshift=-2mm] {\small SPI};
		
		\draw[thick, ->, shorten >=1mm,shorten <=1mm] (reg) -- (led);
		\draw[thick, ->, shorten >=1mm,shorten <=1mm] (reg) -- (7Se);
		\draw[thick, <-, shorten >=1mm,shorten <=1mm] (mcu) -- (taster);	
		
		\draw[thick, <-, shorten <=1mm] (mcu.north) -- (opa.south) node [near start, xshift=3mm, yshift=10mm, rotate=90] {\small analog};
		
		\draw[thick, ->, shorten <=1mm] (L298) -- (shu);	
		\draw[thick, ->, shorten >=1mm] (shu) -- (mot);	
		\draw[thick, ->, shorten >=1mm] (shu) -- (opa);	
		
		\end{tikzpicture}
		\caption{Struktureller Aufbau ...}
	\end{figure}
	%
	\section{Spannungsversorgung}
	Das Board wird mit $0$ bis $\SI{12.5}{\volt}$ \textsmaller{DC} versorgt.
	Eine Diode... .
		%
	\begin{figure}[h!]
		\centering
		%\includegraphics[width = \textwidth]{../images/somefile.pdf}
		\caption{Spannungsversorgung}
	\end{figure}
	%
	%%%%%%%%%%%%%%%%%%%%%%%%%%%%%%%%%%%%%%%%%%%%%%%%
	\section{Vorwiderstände der Leuchtdioden}
	Die \textsmaller{LED}s ...
	Die Vorwiderstände ergeben sich gemäß:
	\begin{equation}
	R_{\text{LED}} = \frac{U_\text{CC} - U_\text{F}}{I_\text{F}}
	\end{equation}
	Die Leistung am Vorwiderstand ergibt sich aus:
	\begin{equation}
	P = (U_\text{CC} - U_\text{F}) \cdot I_\text{F} = R_{\text{LED}} \cdot I_\text{F}^2 
	\end{equation}
	Das Schieberegister hat eine Ausgangsspannung $U_\text{CC} = \SI{5}{\volt}$ und den Ausgangsstrom $I_\text{F} = \SI{20}{\milli\ampere}$.
	Die Vorwiderstände sind in Tabelle~\ref{tab:Rled} aufgeführt.
	%
	\begin{table}[H]
		\caption{Vorwiderstände in Abhängigkeit der Farbe mit der Leistung, die über dem Vorwiderstand abfällt. }
		\label{tab:Rled}
		\centering
		\begin{tabular}{@{}lllll@{}}
		\toprule
			Farbe	& 	$U_\text{LED}$ ($\si{\volt}$)	& 	$R_\text{LED}$ ($\si{\ohm}$)	& 	$R_\text{LED,\,E24}$ ($\si{\ohm}$) 	&	$P_{R_\text{LED}}$ ($\si{\mW}$)		\\
		\midrule
			Rot		&	$\num{1.6}$			&	$\num{170}$		&	$\num{180}$			&	$\num{72}$	\\
			Grün	&	$\num{2.1}$			&	$\num{145}$		&	$\num{150}$			&	$\num{60}$	\\
			Gelb	&	$\num{2.2}$			&	$\num{140}$		&	$\num{150}$			&	$\num{60}$	\\
			Blau	&	$\num{2.9}$			&	$\num{105}$		&	$\num{110}$			&	$\num{44}$	\\
		\bottomrule
		\end{tabular}
	\end{table}
	%
	%%%%%%%%%%%%%%%%%%%%%%%%%%%%%%%%%%%%%%%%%%%%%%%%
	\section{Abschätzung der Leistungsaufnahme}
	...
	%%%%%%%%%%%%%%%%%%%%%%%%%%%%%%%%%%%%%%%%%%%%%%%%
	\section{Entprellung}
	\begin{figure}[h]
		\centering
		%\includegraphics[width = 0.6\textwidth]{../images/somefile.png}
		\caption{Entprellung für einen mechanischen Taster -- fünf mal auf dem Board vorhanden.}
	\end{figure}
	Die Entprellung der mechanischen Taster ...
	
	Zu beachten ist, dass bei geschlossenem Taster ein Strom von $\SI{0.0}{\mA}$ dauerhaft über $R_{1}$ fließt.
	%
	\subsection*{Dimensionierung}
	Der Kondensator wurde mit $C=\SI{0}{\nano\farad}$ und der Widerstand mit $R_1 = \SI{0}{\kilo\ohm}$ gewählt.
	Die Zeitkonstante beträgt damit
	\begin{equation}
		\tau = R \cdot C = \SI{10}{\us}.
	\end{equation}
	%%%%%%%%%%%%%%%%%%%%%%%%%%%%%%%%%%%%%%%%%%%%%%%%
	\section{Ansteuerung des Motors}
	\begin{figure}[h!]
		\centering
		%\includegraphics[width = \textwidth]{../images/...}
		\caption{Motorsteuerung und Strommessung.}\label{Motor}
	\end{figure}
	Die beiden \textsmaller{DC}-Motoren... 
	
	Zur Messung des Motorstroms ... Shunt-Widerstand mit $R=\SI{0.0}{\ohm}$ geleitet.
	... $I=\SI{1}{\A}$ ...
	\begin{equation}
		P=I^2\cdot R=\SI{0.0}{\W}
	\end{equation}
	%%%%%%%%%%%%%%%%%%%%%%%%%%%%%%%%%%%%%%%%%%%%%%%%
	\section{Pinbelegung des STM32 L476RG}
	\begin{figure}[H]
		\centering
		\begin{tikzpicture}[scale=0.40,
		pin/.style={draw,rectangle,minimum width=1.5em,font=\tiny, rounded corners=1pt, fill=white}
		]
		\draw[ultra thick, rounded corners=2pt] (0,0) rectangle (23.3,23.3);
		\draw[rounded corners=1pt] (0.4,17.6) rectangle (0.8,17.2);
		\draw[rounded corners=1pt, dashed] (2,2) rectangle (21.3,21.3);
		\node at (11.565,11.565){STM32 L476};
		
		% Main trick: loop over the label numbers and then adjust their position
		% in the tikzpicture using evaluate to calculate \y=y-coordinate of pin
		\foreach \i/\desc [evaluate=\i as \y using (21.25-(\i-1)*1.3)]
		in {1/{VBAT},
			2/{PC13},
			3/{PC14},
			4/{PC15},
			5/{PH0},
			6/{PH1},
			7/{NRST},
			8/{PCO},
			9/{PC1},
			10/{PC2},
			11/{PC3},
			12/{VSS},
			13/{VDD},
			14/{PA0},
			15/{PA1},
			16/{PA2}}%
		{
			\draw node[pin,anchor=east] at (1.2,\y){$\i$};
			\node[align=right,anchor=east] at (-0.4,\y){\footnotesize\desc};
		}
		\foreach \i/\desc [evaluate=\i as \x using (1.5+(\i-17)) * 1.3]
		in {17/{PA3},
			18/{VSS},
			19/{VDD},
			20/{PA4},
			21/{PA5},
			22/{PA6},
			23/{PA7},
			24/{PC4},
			25/{PC5},
			26/{PB0},
			27/{PB1},
			28/{PB2},
			29/{PB10},
			30/{PB11},
			31/{VSS},
			32/{VDD}%
		}
		{
			\draw node[pin,anchor=east,rotate=90] at (\x,1.2){$\i$};
			\node[align=right,anchor=east,rotate=90] at (\x,-0.4){\footnotesize\desc};
		}
		\foreach \i/\desc [evaluate=\i as \y using ((\i-33.15) * 1.3)+2]
		in {33/{PB12},
			34/{PB13},
			35/{PB14},
			36/{PB15},
			37/{PC6},
			38/{PC7},
			39/{PC8},
			40/{PC9},
			41/{PA8},
			42/{PA9},
			43/{PA10},
			44/{PA11},
			45/{PA12},
			46/{PA13},
			47/{VSS},
			48/{VDD}%
		}
		{
			\draw node[pin,anchor=west] at (22.1,\y){$\i$};
			\node[align=left,anchor=west] at (23.7,\y){\footnotesize\desc};
		}
		\foreach \i/\desc [evaluate=\i as \x using (25-(\i - 46.2)*1.3)]
		in {49/{PA14},
			50/{PA15},
			51/{PC10},
			52/{PC11},
			53/{PC12},
			54/{PD2},
			55/{PB3},
			56/{PB4},
			57/{PB5},
			58/{PB6},
			59/{PB7},
			60/{BOOT0},
			61/{PB8},
			62/{PB9},
			63/{VSS},
			64/{VDD}%
		}
		{
			\draw node[pin,anchor=west,rotate=90] at (\x,22.1){$\i$};
			\node[align=right,anchor=west,rotate=90] at (\x,23.7){\footnotesize\desc};
		}
		\end{tikzpicture}
		\caption{Pinbelegung}
	\end{figure}
\begin{table}[H]
	\caption{Pinbelegung des STM32 L476}
	\renewcommand{\arraystretch}{1.0}
	\centering
	\begin{minipage}{.48\linewidth}
		\begin{tabularx}{\linewidth}{@{}rllX@{}}
			\toprule
			Nr. & Name & Funktion \\
			\midrule
			1 	& VBAT\\
			2 	& PC13 & B1 [Blue PushButton]\\
			3 	& PC14 \\
			4 	& PC15 \\
			5 	& PH0 \\
			6 	& PH1 \\
			7 	& NRST & \\
			8 	& PC0 & \\
			9 	& PC1 & \\ 
			10	& PC2 & \\
			11	& PC3 & \\
			12  & VSS & \\
			13  & VDD & \\
			14  & PA0 & \\
			15  & PA1 & \\
			16  & PA2 & \\ 
			\midrule
			17  & PA3  & \\
			18  & VSS \\ 
			19  & VDD & \\ 
			20  & PA4 & \\ 
			21  & PA5 & LD2 [green Led] \\ 
			22  & PA6 \\ 
			23  & PA7 & \\
			24  & PC4 & \\ 
			25  & PC5 & \\ 
			26  & PB0  & \\ 
			27  & PB1 \\ 
			28  & PB2\\ 
			29  & PB10 & \\
			30  & PB11 & \\
			31  & VSS & \\
			32  & VDD & \\			
			\bottomrule
		\end{tabularx}
	\end{minipage}%
	\hfill
	\begin{minipage}{.48\linewidth}
		\begin{tabularx}{\linewidth}{@{}rllX@{}}
			\toprule
			Nr. & Name & Funktion \\
			\midrule
			33 	& PB12 \\
			34 	& PB13 &  \\
			35 	& PB14 &  \\
			36	& PB15 & \\
			37 	& PC6 & \\
			38 	& PC7 & \\
			39 	& PC8 & \\
			40 	& PC9 & \\
			41 	& PA8 &  \\ 
			42	& PA9 &  \\
			43	& PA10 & \\
			44  & PA11\\
			45  & PA12 & \\
			46  & PA13 & \\
			47  & VSS & \\
			48  & VDD & \\ 
			\midrule
			49  & PA14 & \\
			50  & PA15 & \\ 
			51  & PC10 & \\ 
			52  & PC11 & \\ 
			53  & PC12 & \\ 
			54  & PD2 & \\ 
			55  & PB3 & \\
			56  & PB4 & \\ 
			57  & PB5 & \\ 
			58  & PB6\\ 
			59  & PB7 & \\ 
			60  & BOOT0 \\
			61  & PB8 & \\
			62  & PB9 & \\
			63  & VSS & \\
			64  & VDD & \\ 			
			\bottomrule
		\end{tabularx}%
	\end{minipage}
\end{table}
%%%%%%%%%%%%%%%%%%%%%%%%%%%%%%%%%%%%%%%%%%%%%%%%
	\section{Layout}
	\begin{figure}[h]
		\centering
		%\includegraphics[]{../images/topView.pdf}
		\caption{Top-Layer des Boards von oben}
	\end{figure}
	\begin{figure}[h]
		\centering
		%\includegraphics[]{../images/bottomView.pdf}
		\caption{Bottom-Layer des Boards von oben}
	\end{figure}
	\begin{figure}[h]
		\centering
		%\includegraphics[]{../images/topParts.pdf}
		\caption{Board mit Bestückung von oben}
	\end{figure}
%
%%%%%%%%%%%%%%%%%%%%%%%%%%%%%%%%%%%%%%%%%%%%%%%%
\chapter{Software}

\section{Aufbau}
	\begin{figure}[h]
		\centering
		%\includegraphics[width = 0.8\linewidth]{../images/softwarestructure.pdf}
		\caption{Struktureller Aufbau der Software}
		\label{fig:AufbauSoftware}
	\end{figure}
Zur Implementierung ...
%%%%%%%%%%%%%%%%%%%%%%%%%%%%%%%%%%%%%%%%%%%%%%%%
\section{Zustandsmaschine}
Am besten in Form einer Tabelle:
\begin{table}[H]
\centering
\caption{Zustandsübergänge...}
\begin{tabular}{@{}lll@{}}
\toprule
Zustand & Ereignis & Folgezustand \\
\midrule
\textsmaller{PREINIT, ERROR} & \textsmaller{CAN}-Msg. Init. & \textsmaller{ACTIVE}.init\\
\cmidrule{1-3}
\textsmaller{ACTIVE}& Overcurrent \& .grabbing & \textsmaller{IDLE}\\
& Overcurrent \& .moving & \textsmaller{ERROR}\\
& Undercurrent  & \textsmaller{ERROR}\\
...\\
\cmidrule{1-3}
\textsmaller{IDLE, ACTIVE}& \textsmaller{CAN}-Msg. MoveRel & \textsmaller{ACTIVE}.moving\\
& \textsmaller{CAN}-Msg. MoveAbs & \textsmaller{ACTIVE}.moving\\
...\\
\bottomrule
\end{tabular}
\end{table}
%%%%%%%%%%%%%%%%%%%%%%%%%%%%%%%%%%%%%%%%%%%%%%%%
\section{Quellcode in \LaTeX{}}
\subsection{So geht's}
Die \texttt{main}-Funktion ...
\begin{lstlisting}[style=myC]
// SPI
Board_initSPI();

SPI_Params_init(&spi_params);
spi_params.transferMode = SPI_MODE_BLOCKING;
spi_params.bitRate  = 4000000;
spi_params.mode        = SPI_MASTER;
spi_params.dataSize = 16;
spi_params.transferTimeout = SPI_WAIT_FOREVER;
spi_params.transferCallbackFxn = NULL;
spi_handle = SPI_open(Board_SPI0, &spi_params);
if (spi_handle) {
    System_printf("SPI opened\r\n");
} else {
    System_printf("SPI did not open\r\n");
}
\end{lstlisting}
%%%%%%%%%%%%%%%%%%%%%%%%%%%%%%%%%%%%%%%%%%%%%%%%
\chapter{Fazit}
...
%%%%%%%%%%%%%%%%%%%%%%%%%%%%%%%%%%%%%%%%%%%%%%%%
\chapter{Errata}
	Folgende Fehler:
	\begin{itemize}
		\item xxx
		\item yyy
		\item zzz
	\end{itemize}
%%%%%%%%%%%%%%%%%%%%%%%%%%%%%%%%%%%%%%%%%%%%%%%%
\end{document}
