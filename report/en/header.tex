%\usepackage[utf8]{inputenc}
\usepackage[T1]{fontenc}
\usepackage[  
    vmarginratio=1:2, %Verhältnis der oben/unten Seitenränder zur automatischen Berechnung
    paper=a4paper,
    %left=3cm,
    %right=3cm,
    inner=3.5cm, % mittlerer Rand
    outer=2.5cm, % äußerer Rand
    top=2.0cm,
    bottom=2.5cm,
    %marginparwidth=2.3cm, % Breite des Marginpars
    includeheadfoot, % Kopfzeile in Berechnung einbeziehen
    %includemp % Marginpar in die Berechnung mit einbeziehen
    %showframe=true,
]{geometry}
\usepackage{tabularx}
\usepackage{booktabs}
\usepackage[english]{babel}
\usepackage{mathpazo}
\usepackage[euler-digits]{eulervm}

\usepackage[no-math]{fontspec} %no-math
%\usepackage{fontspec}
\usepackage{amsmath}
%\usepackage[math-style=upright]{unicode-math}
%\setmathfont{TeX Gyre Pagella Math}
%\setmathfont{Neo Euler}
%\%setmathfont[math-style=ISO,bold-style=ISO,vargreek-shape=TeX]{Libertinus Math}
%\defaultfontfeatures{SmallCapsFeatures={Renderer=Basic}}
\setsansfont[Path =../fonts/,
UprightFont = *-Regular,
BoldFont = *-Bold,
ItalicFont = *-Italic,
BoldItalicFont = *-BoldItalic
]{URWClassico}
 
%\setmainfont[Path =./fonts/,
%UprightFont = *-regular,
%BoldFont = *-bold,
%ItalicFont = *-italic,
%BoldItalicFont = *-bolditalic]{texgyrepagella}

\setmainfont[Path =../fonts/,
UprightFont = *,
BoldFont = *b,
ItalicFont = *i,
BoldItalicFont = *bi]{pala}

\setmonofont[Path = ../fonts/,
UprightFont = *-Regular,
BoldFont = *-Bold,
AutoFakeSlant,
BoldItalicFeatures={FakeSlant}
]{Inconsolata}

\usepackage{MnSymbol}

\usepackage[tracking=true]{microtype}
\SetTracking[no ligatures]{encoding = *, shape = sc*}{150}
\usepackage[
pdfencoding=auto,
colorlinks=true,
linkcolor=black,
citecolor=black,
filecolor=black,
urlcolor=black,
bookmarks=true,
bookmarksopen=true,
bookmarksopenlevel=3,
plainpages=false,
pdfpagelabels=true,
pdftitle={Praktikum Mikrocontroller},
pdfauthor={},
pdfsubject={}, % subject of the document
pdfkeywords={}]{hyperref}
\usepackage{mathtools}
\usepackage{setspace}
\spacing{1.2}
\usepackage{enumitem}
%\usepackage{rotating}
\usepackage{booktabs}
\usepackage{pgfplots}
\usetikzlibrary{pgfplots.groupplots} 
\usetikzlibrary{plotmarks}
\usepackage{tikz-3dplot}
\usetikzlibrary{calc}
\usepackage{polynom}
\polyset{style=C, div=:,vars=x}
\usepackage{wrapfig}
\usepackage[locale=US]{siunitx}
\DeclareSIUnit{\promille}{\text{\textperthousand}}
%\DeclareSIUnit{\percent}{\text{\textpercent}}
\DeclareSIUnit{\dBm}{dBm}
\sisetup{
	math-rm=\mathrm,
	mode=math,
	unitmode=text,
    %detect-all,               %% Benutze gleiche Schriftarten wie im Text
    group-digits=true,          %% Zifferngruppierung an/aus
    group-separator=\, ,        %% Zeichen für Zifferngruppierung
    group-minimum-digits=5     %% Ziffern ab # Ziffern gruppieren
}
\usepackage{relsize}
\usepackage{cancel}
\usepackage{nicefrac}
\usepackage{xcolor}
\usepackage{tikz}
\usepackage{lscape}
\usepackage{here}
\usepackage{selnolig}
%\nolig{Haupttask}{Haupt|task}
%\nolig{Starttemperatur}{Start|temperatur}
\usepackage{scrpage2} 
\setheadsepline{0.5pt}
\clearscrheadings
\clearscrplain                  
\pagestyle{scrheadings}
\automark[section]{chapter}
%Nun die Befehle für die Kopf- und Fusszeileneinstellung. Der Befehl ist immer für beide Seitenstile (plain,headings) und hat immer das Format:
%\Position[scrplain]{scrheadings}
\ohead[]{\rightmark}
\ofoot[\pagemark]{\pagemark}
\renewcommand*{\headfont}{\normalfont\sffamily}
\renewcommand*{\footfont}{\normalfont\sffamily}
\renewcommand*{\pnumfont}{\normalfont\sffamily}

%\usepackage{mathtools}
\usetikzlibrary{positioning,shadows,backgrounds}
% Wurzel mit Haken hinten (Dieser Hack hat eine Größenordnung nationaler Verschuldung)
\usepackage{letltxmacro}
\LetLtxMacro{\oldsqrt}{\sqrt}
\renewcommand\sqrt[2][]{\mathpalette\DHLhksqrtA{{#1}{#2}}}
\def\DHLhksqrtA#1#2{\setbox0=\hbox{$#1\DHLhksqrtB#2$}\dimen0=\ht0
\advance\dimen0-0.15\ht0
%0.15 ist das Mass fuer die Hakenlaenge, relativ zum Inhalt der Wurzel
\setbox2=\hbox{\vrule height 0.98\ht0 depth -\dimen0}%
{\box0\lower0.4pt\box2}}
\def\DHLhksqrtB#1#2{\def\a{#1}\def\b{}\ifx\a\b\oldsqrt{#2\,}\else\oldsqrt[#1]{#2\,}\fi}

\newcommand{\remark}[1]{{\color{red}#1}}

\usepackage[tocindentauto]{tocstyle}
\usetocstyle{KOMAlike}

\newsavebox{\myendhook}  % for the tabulars
\makeatletter
\def\tagform@#1{{(\maketag@@@{\ignorespaces#1\unskip\@@italiccorr)}%
		\makebox[0pt][r]{%  after  the  equation  number
			\makebox[0.4\textwidth][l]{\usebox{\myendhook}}%
		}%
		\global\sbox{\myendhook}{}%  clear  box  content
	}%
}
\makeatother

\usepackage{ncccomma} %Komma als Dezimaltrennzeichen
\usepackage{url}

\addtokomafont{disposition}{\boldmath}
\newcommand{\abs}[1]{\ensuremath{\left\vert#1\right\vert}}
\usepackage[europeanresistors]{circuitikz}

\definecolor{dkgreen}{rgb}{0,0.6,0}
\definecolor{gray}{rgb}{0.5,0.5,0.5}
\definecolor{mauve}{rgb}{0.58,0,0.82}
\usepackage{listings}
\lstdefinestyle{myC}{ %
	columns=fullflexible,
  language=C,                  % the language of the code
  basicstyle=\footnotesize\ttfamily\color{black},       % the size of the fonts that are used for the code
  numbers=left,                   % where to put the line-numbers
  numberstyle=\tiny\color{black},  % the style that is used for the line-numbers
  stepnumber=5,firstnumber=1,numberfirstline,
                                  % will be numbered
  numbersep=5pt,                  % how far the line-numbers are from the code
  backgroundcolor=\color{white},  % choose the background color. You must add \usepackage{color}
  showspaces=false,               % show spaces adding particular underscores
  showstringspaces=false,         % underline spaces within strings
  showtabs=false,                 % show tabs within strings adding particular underscores
  frame=none,                   % adds a frame around the code
  rulecolor=\color{black},        % if not set, the frame-color may be changed on line-breaks within not-black text (e.g. commens (green here))
  tabsize=4,                      % sets default tabsize to 2 spaces
  captionpos=b,                   % sets the caption-position to bottom
  breaklines=true,                % sets automatic line breaking
  breakatwhitespace=true,        % sets if automatic breaks should only happen at whitespace
  title=\lstname,                 % show the filename of files included with \lstinputlisting;
                                  % also try caption instead of title
  keywordstyle=\bfseries\color{black},          % keyword style
  commentstyle=\itshape\color{gray},       % comment style
  stringstyle=\color{gray},         % string literal style
  escapeinside={\%*}{*)},            % if you want to add a comment within your code
  morekeywords={uint8_t,uint16_t,uint32_t},               % if you want to add more keywords to the set
  lineskip={1pt}
}
